\documentclass[main.tex,fontsize=8pt,paper=a4,paper=portrait,DIV=calc,]{scrartcl}
% Document
\usepackage[T1]{fontenc}
\usepackage[utf8]{inputenc}
\usepackage[dvipsnames]{xcolor}
\usepackage[english]{babel} 
\usepackage{hyperref}
\usepackage{sectsty}
\usepackage[final]{pdfpages}
\renewcommand{\familydefault}{\sfdefault}

% Format
\usepackage[top=5mm,bottom=5mm,left=5mm,right=5mm]{geometry}
\setlength{\headheight}{3\baselineskip}
\setlength{\headsep}{2mm}

\usepackage{scrlayer-scrpage}
\clearpairofpagestyles
\chead{{\bfseries\TITLE, \AUTHOR, \pagename~\thepage}}

% \usepackage{multicol}
% \setlength{\columnsep}{2mm}
% \setlength{\columnseprule}{0.1pt}

% Math
\usepackage{amsmath}
\usepackage{amssymb}
\usepackage{amsfonts}

% Code
\usepackage{fancyvrb, etoolbox, listings, xcolor}
    \definecolor{lightgray}{rgb}{0.95, 0.95, 0.95}
    \definecolor{darkgray}{rgb}{0.4, 0.4, 0.4}
    \definecolor{purple}{rgb}{0.65, 0.12, 0.82}
    \definecolor{ocherCode}{rgb}{1, 0.5, 0} % #FF7F00 -> rgb(239, 169, 0)
    \definecolor{blueCode}{rgb}{0, 0, 0.93} % #0000EE -> rgb(0, 0, 238)
    \definecolor{greenCode}{rgb}{0, 0.6, 0} % #009900 -> rgb(0, 153, 0)
    \definecolor{teal}{rgb}{0.0, 0.5, 0.5}

\lstdefinestyle{code}{
    identifierstyle=\color{black},
    keywordstyle=\color{blue}\bfseries\small,
    ndkeywordstyle=\color{greenCode}\bfseries\small,
    stringstyle=\color{ocherCode}\ttfamily\small,
    commentstyle=\color{teal}\ttfamily\textit\small,
    basicstyle=\ttfamily\small,
    breakatwhitespace=false,         
    breaklines=true,                 
    captionpos=b,                    
    keepspaces=true,                 
    showspaces=false,                
    showstringspaces=false,
    showtabs=false,                  
    tabsize=2,
}

% Images
\usepackage{graphicx}
\newcommand{\pic}{\includegraphics[scale=0.3]}
\graphicspath{{Screenshots/}{../Screenshots}}
\makeatletter
\def\pictext#1#2{%
    \@ifnextchar[{%
    \pictext@iiiii{#1}{#2}%
    }{%
      \pictext@iiiii{#1}{#2}[0.5,0.4,0.3]% Default is 5
    }%
}
\def\pictext@iiiii#1#2[#3,#4,#5]{\begin{minipage}{#3\textwidth}\includegraphics[scale=#4]{#1}\end{minipage}\begin{minipage}{#5\textwidth}#2\end{minipage}}
\def\minipg#1#2{%
    \@ifnextchar[{%
    \minipg@iiii{#1}{#2}%
    }{%
      \minipg@iiii{#1}{#2}[0.3,0.6]% Default is 5
    }%
}
\def\minipg@iiii#1#2[#3,#4]{\vspace{0.8mm}\begin{minipage}{#3\textwidth}#1\end{minipage}\begin{minipage}{#4\textwidth}#2\end{minipage}{\vspace{0.8mm}}}
\makeatother

% Smaller Lists
\usepackage{enumitem}
\setlist[itemize,enumerate]{leftmargin=3mm, labelindent=0mm, labelwidth=1mm, labelsep=1mm, nosep}
\setlist[description]{leftmargin=0mm, nosep}
\setlength{\parindent}{0cm}

% Smaller Titles
\usepackage[explicit]{titlesec}

%% Color Boxes
\newcommand{\sectioncolor}[1]{\colorbox{black!60}{\parbox{0.989\linewidth}{\color{white}#1}}}
\newcommand{\subsectioncolor}[1]{\colorbox{black!50}{\parbox{0.989\linewidth}{\color{white}#1}}}
\newcommand{\subsubsectioncolor}[1]{\colorbox{black!40}{\parbox{0.989\linewidth}{\color{white}#1}}}
\newcommand{\paragraphcolor}[1]{\colorbox{black!30}{\parbox{0.989\linewidth}{\color{white}#1}}}
\newcommand{\subparagraphcolor}[1]{\colorbox{black!20}{\parbox{0.989\linewidth}{\color{white}#1}}}

%% Title Format
\titleformat{\section}{\vspace{0.5mm}\bfseries}{}{0mm}{\sectioncolor{\thesection~#1}}[{\vspace{0.5mm}}]
\titleformat{\subsection}{\vspace{0.5mm}\bfseries}{}{0mm}{\subsectioncolor{\thesubsection~#1}}[{\vspace{0.5mm}}]
\titleformat{\subsubsection}{\vspace{0.5mm}\bfseries}{}{0mm}{\subsubsectioncolor{\thesubsubsection~#1}}[{\vspace{0.5mm}}]
\titleformat{\paragraph}{\vspace{0.5mm}\bfseries}{}{0mm}{\paragraphcolor{\theparagraph~#1}}[{\vspace{0.5mm}}]
\titleformat{\subparagraph}{\vspace{0.5mm}\bfseries}{}{0mm}{\subparagraphcolor{\thesubparagraph~#1}}[{\vspace{0.5mm}}]

%% Title Spacing
\titlespacing{\section}{0mm}{0mm}{0mm}
\titlespacing{\subsection}{0mm}{0mm}{0mm}
\titlespacing{\subsubsection}{0mm}{0mm}{0mm}
\titlespacing{\paragraph}{0mm}{0mm}{0mm}
\titlespacing{\subparagraph}{0mm}{0mm}{0mm}

%% format cells
\usepackage[document]{ragged2e}
\usepackage{array, makecell}
\renewcommand{\arraystretch}{2}
\newcommand{\mc}{\makecell[{{m{1\linewidth}}}]}



\lstset{
    language={[x86masm]Assembler},
    style=code,
}

\begin{document}
\tableofcontents

\newcommand{\TITLE}{Secure Software}
\newcommand{\AUTHOR}{Fabio Lenherr}
\setcounter{tocdepth}{1}

\section{Secure Software Principles}
\textcolor{Cyan}{Rather than trying to solve objectives in cyber security, you should have expectations of your software in terms of security.}

\subsection{CVSS Common Vulnerability Scoring System}
\begin{itemize}
  \item \textcolor{red}{High}\newline
    Heartbleed, log4j etc\newline
    often not noticeable, concealed
  \item \textcolor{blue}{Medium}\newline
    unclear if it can be exploited
  \item \textcolor{green}{Low}\newline
    typical problems like outdated tls certificate
\end{itemize}

\subsection{Low-hanging fruits first}
\textcolor{teal}{Before we go ahead and tackle the biggest issues, we can solve a lot of problems by making sure that we do not fail at the basics.}\newline
In other words, first try to solve the easy things.

\subsection{80\%/20\% Problem}
\begin{itemize}
\item \textcolor{purple}{Secure the weakest link}\newline
  A boomer might not know about phishing attacks, protect said user against doing something dumb!
\item \textcolor{purple}{Practice defense in depth}\newline
  Use multiple security layers, often 1 is not enough
\item \textcolor{purple}{Fail secure}\newline
  In terms of an error, don't just dump all information to some random person, eg. don't leak credit card information if the information is not 100\% correct.
\item \textcolor{purple}{Follow the principle of least privilege}\newline
  Don't give random people access to things they don't need
\item \textcolor{purple}{To Compartmentalize}\newline
  Try to categorize tasks, this makes it easier for people to get access to something.
\item \textcolor{purple}{Keep it simple}\newline
  Try to keep it as simple as possible
\item \textcolor{purple}{Promote privacy}\newline
  Handle privacy of users appropriately
\item \textcolor{purple}{Remember that hiding secrets is hard}
  In other words, security by obscurity is a novel practice and doesn't really work
\item \textcolor{purple}{Be reluctant to trust}
\item \textcolor{purple}{Use your community resources}
\end{itemize}

\section{Threat Models \& Mitigations}
\subsection{The 4 Questions}
\begin{enumerate}
\item \textcolor{purple}{What are we working on?}\newline
  \begin{itemize}
  \item \textcolor{black}{Define scope and context}
  \item \textcolor{black}{Description of requirements and design}
  \end{itemize} 
\item \textcolor{purple}{What can go wrong?}\newline
  Anciticapte potential issues (think like the attacker)
\item \textcolor{purple}{What are we going to do about it?}\newline
  Define mitigations
\item \textcolor{purple}{Did we do a good job?}\newline
  Reflection and confirmation
\end{enumerate} 

\subsection{Methodology for Threat Risk Modeling}
\begin{enumerate}
\item \textcolor{purple}{Identify security objectives with a focus on:}\newline
 \begin{itemize}
 \item \textcolor{black}{sensitive information stored on device}
 \item \textcolor{black}{third party libraries used}
 \item \textcolor{black}{loss of repudation derived from misuse of the application}
 \end{itemize} 
\item \textcolor{purple}{Break down the application features, i.e. decomposition of the application into small feature sets with the goal to identify security impact}
\item \textcolor{purple}{Identification of related threats and vulnerabilities in implementation}
\end{enumerate} 

\subsection{Process Steps}
\begin{enumerate}
\item \textcolor{purple}{Identity Assets}\newline
What do we own and what do we want to protect? Not everything is worth protecting!
\item \textcolor{purple}{Create an architecture overview}\newline
  Simple diagrams which include application, subsystems, trust boundaries and data flow
\item \textcolor{purple}{Decompose the application}\newline
  The structure of your application -> network model, API, etc that can lead to potential vulnerabilities
\item \textcolor{purple}{Identify the threats}\newline
  From the attackers perspective, try to find loopholes etc to attack your application and \emph{document them}
\item \textcolor{purple}{Document the threats}\newline
\item \textcolor{purple}{Rate the threats} \newline
  Rate the threats according to \emph{likelihood, impact damage (reputation, cost), damage of migitation (cost, annoyance or work)}
\end{enumerate} 

\subsection{Why?}
The reason we do all this is that the following 3 groups have the these benefits: 
\begin{itemize}
\item \textcolor{purple}{Developers}\newline
  \begin{itemize}
  \item \textcolor{black}{Know potential threats and develop with them in mind}
  \item \textcolor{black}{priorization of threats to proper mitigations can be taken}
  \end{itemize} 
\item \textcolor{purple}{Penetration Testers}\newline
  \begin{itemize}
  \item \textcolor{black}{Makes sure penetration testers don't waste time testing things that don't matter to you}
  \item \textcolor{black}{Have a decent idea where to start}
  \end{itemize} 
\item \textcolor{purple}{Management}\newline
  \begin{itemize}
  \item \textcolor{black}{Know the risks in their language -> cost}
  \item \textcolor{black}{Don't make stupid decisions against developers etc since they don't understand IT}
  \end{itemize} 
\end{itemize} 

\subsection{STRIDE}
\begin{itemize}
\item \textcolor{purple}{Spoofing}\newline
Using a false identity to gain access to a system.\newline
\textcolor{green}{Mitigation: Authentication} Cookies, Signatures, updates TLS
\item \textcolor{purple}{Tampering}\newline
unauthorized changes/manipulation of data\newline
\textcolor{green}{Mitigation: Integrity} ACLs, Digital Signatures
\item \textcolor{purple}{Repudation}\newline
The ability to deny having performed an attack,\newline
\textcolor{green}{Mitigation: Nonrepudation} Secure logging and auditing, Digital Signatures
\item \textcolor{purple}{Impersonation}\newline
unauthorized revelation of classified(PROPRIETARY) private information.\newline
\textcolor{green}{Mitigation: Confidentiality} Encryption, ACLS
\item \textcolor{purple}{Denial of Service}\newline
Prevent or restrict access to a service by flooding it.\newline
\textcolor{green}{Mitigation: Availability} ACLs, Filtering, Quotas
\item \textcolor{purple}{Elevation of Privilege}\newline
Gaining unauthorized privileges on a system. For example: becoming root as regular user\newline
\textcolor{green}{Mitigation: Authorization} ACLs, Group or role membership, privilege ownership, input validation
\end{itemize} 

\subsubsection{Stride and Thread Modeling}
\includegraphics[scale=0.4]{2023_03_02_10_46_31.png}

\subsection{Attack Tree}
This is a list of attacks that can lead to another. This can help with attacks that are \emph{not obvious}.\newline
\textcolor{puple}{Attack trees are complementary to threat modeling}\newline
\textcolor{red}{Here an example of how they work:}\newline
\includegraphics[scale=0.4]{2023_03_02_11_20_47.png}
\includegraphics[scale=0.4]{2023_03_02_11_22_00.png}\newline
\textcolor{purple}{Add labels for possible and impossible actions}\newline
\includegraphics[scale=0.4]{2023_03_02_11_22_15.png}
\includegraphics[scale=0.4]{2023_03_02_11_22_25.png}\newline
\textcolor{purple}{Propagate the labels up so possible things are closer, after that \emph{define used tools for the actoins}}\newline
\includegraphics[scale=0.4]{2023_03_02_11_23_15.png}
\includegraphics[scale=0.4]{2023_03_02_11_23_34.png}\newline
\textcolor{purple}{Define monetary costs for the attack and \emph{mark the route with least cost and the least tools used}}

\subsection{Risk Management Acitivies}
\includegraphics[scale=0.4]{2023_03_02_11_13_47.png}\newline

\section{Software bugs and unexpected behavior}
\subsection{Types of bugs and issues}
\begin{itemize}
\item \textcolor{purple}{security bug}\newline
These are a misimplementation and are easy to discover and remediate using modern code review tools.\newline
Examples include buffer overflows, race conditions and unsafe system calls
\item \textcolor{purple}{security flaw}\newline
This is a design flaw and is hence impossible to detect by automated tools. \newline
This might be bad error handling, type confusion, or just plain wrong usage of something
\item \textcolor{purple}{Decurity defect: security bug + security flaw}
\end{itemize} 

\subsection{Buffer overflow}
\textcolor{purple}{Buffer overflow attacks are common in the C language space. Mostly this is because of use of legacy functions that fail to check for safety.\newline
This can lead to examples where you would like to check a password, however, the attacker provides not just a password, but something longer. When comparing with unsafe code, this could lead to the actual password being overwritten and the attacker therefore gaining access even with a faulty password.}\newline
Here a concrete example for this:
\begin{lstlisting}
#include <stdio.h>
#include <string.h>
void printstr(char *str, int len) {
  for (int i = 0; i < len; i++) {
    printf("%c", str[i]);
  }
  printf("\n");
}

void printstr_unsafe(char *str) {
  int i = 0;
  while (str[i] != '\0') {
    printf("%c", str[i]);
    i++;
  }
  printf("\n");
}

int compstr(char* str1, char* str2, int lower) {
  for(int i = 0; i < lower; i++) {
    // printf("%c", str1[i]);
    // printf("%c", str2[i]);
    if(str1[i] != str2[i]) {
      return 0;
    }
  }
  return 1;
}

int main(int argc, char **argv) {
  int a = 5;
  int b = 7;
  int c = a + b;

  char arr[] = {'g', 'e', 't', 'm', 'e', 'h', 'a', 'c', 'k', 'e', 'd'};
  char arr3[8];
  char arr2[] = "pwsecret";
  
  strcpy(arr3, argv[1]);
  if (compstr(arr2, arr3, 8) == 1) {
    printstr(arr, sizeof(arr));
    return 1;
  }

  printf("you fucked up\n");
  // printstr(arr2, sizeof(arr2));
  // printstr_unsafe(arr2);
  printstr(arr3, sizeof(arr3));   
  printstr(arr2, sizeof(arr2));   

  return 0;
}
\end{lstlisting}
\textcolor{teal}{Note that if the attacker overdoes the length of the malicious string, then the program will read into unallocated memory. This will lead to the OS segfaulting this program.}\newline
\textcolor{red}{Solution: Use modern and secure functions for C/C++ OR use more modern languages such as rust.}

\subsection{Image Visualization of Stack}
\includegraphics[scale=0.4]{2023_03_16_11_27_16.png}
\includegraphics[scale=0.4]{2023_03_16_11_27_42.png}

\subsection{Image Visualization of an attack}
\includegraphics[scale=0.4]{2023_03_16_11_28_24.png}\newline
\textcolor{red}{However, you can just simply create a different return location, since we can override the return address :)}\newline
For this we first have to choose a function to use, let's take execve in order to replace the current program with a shell.\newline
\includegraphics[scale=0.4]{2023_03_16_11_31_03.png}\newline
Then we have to create a main function with a call to execve in order to be able to jump to it.\newline
\textcolor{teal}{Even more important, the function then needs to be disassembled and formatted in order to be used.}\newline
\includegraphics[scale=0.4]{2023_03_16_11_31_59.png}\newline
We can now start to override the return function with this new text.\newline
\includegraphics[scale=0.4]{2023_03_16_11_33_05.png}

\subsubsection{NOP}
\textcolor{purple}{The problem we face with return addresses, is that we do not know the exact address that we need to call. This means that we might miss the actual address of our code.\newline
Keep in mind that our address is just offset from the stackpointer. }\newline
\textcolor{teal}{However, there is a solution, we can use NOP instructions as padding around the code. This means that if we jump too high, we will simply run into these NOP statements, which do nothing other than telling the cpu to move to the next statement below.\newline
We can simply repeat this until we are at the evil code.\newline
Should we land below the evil code, then we can use ret statements with an address that is prob above the evil code to run it.}


\section{Web Security}

\section{Reverse Engineering}

\section{Secure Software Lifecycle}

\section{Mobile Security}

\section{Security Testing}

\section{Bug Bounty}

\section{Software Security Assurance}

\section{Summary}

\end{document}
