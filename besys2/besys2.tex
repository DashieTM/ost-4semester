\documentclass[main.tex,fontsize=8pt,paper=a4,paper=portrait,DIV=calc,]{scrartcl}
% Document
\usepackage[T1]{fontenc}
\usepackage[utf8]{inputenc}
\usepackage[dvipsnames]{xcolor}
\usepackage[nswissgerman,english]{babel} 
\usepackage{hyperref}
\renewcommand{\familydefault}{\sfdefault}

% Format
\usepackage[top=5mm,bottom=1mm,left=5mm,right=5mm]{geometry}
%\setlength{\headheight}{\baselineskip}
%\setlength{\headsep}{0mm}

%\usepackage{scrlayer-scrpage}
%\clearpairofpagestyles
%\chead{{\bfseries\TITLE, \AUTHOR, \pagename~\thepage}}

%\addtokomafont{pagehead}{\upshape}

\usepackage{multicol}
\setlength{\columnsep}{2mm}
\setlength{\columnseprule}{0.1pt}

% Math
\usepackage{amsmath}
\usepackage{amssymb}
\usepackage{amsfonts}

% Code
\usepackage{fancyvrb, etoolbox, listings, xcolor}
%\usemintedstyle{bw}

%\newminted[shell]{bash}{
%fontsize=\footnotesize,
%fontfamily=tt,
%breaklines=true,
%frame=single,
%framerule=0.1pt,
%framesep=2mm,
%tabsize=2
%}
%\newminted{css}{
%breaklines=true,
%tabsize=4,
%autogobble=true,
%escapeinside=||,
%stripall=true,
%stripnl=true,
%}

    \definecolor{lightgray}{rgb}{0.95, 0.95, 0.95}
    \definecolor{darkgray}{rgb}{0.4, 0.4, 0.4}
    \definecolor{purple}{rgb}{0.65, 0.12, 0.82}
    \definecolor{ocherCode}{rgb}{1, 0.5, 0} % #FF7F00 -> rgb(239, 169, 0)
    \definecolor{blueCode}{rgb}{0, 0, 0.93} % #0000EE -> rgb(0, 0, 238)
    \definecolor{greenCode}{rgb}{0, 0.6, 0} % #009900 -> rgb(0, 153, 0)
    \definecolor{teal}{rgb}{0.0, 0.5, 0.5}

\lstdefinestyle{code}{
    identifierstyle=\color{black},
    keywordstyle=\color{blue}\bfseries\small,
    ndkeywordstyle=\color{greenCode}\bfseries\small,
    stringstyle=\color{ocherCode}\ttfamily\small,
    commentstyle=\color{teal}\ttfamily\textit\small,
    basicstyle=\ttfamily\small,
    breakatwhitespace=false,         
    breaklines=true,                 
    captionpos=b,                    
    keepspaces=true,                 
    showspaces=false,                
    showstringspaces=false,
    showtabs=false,                  
    tabsize=2,
    belowskip=-5pt
}



% Images
\usepackage{graphicx}
\newcommand{\pic}{\includegraphics[scale=0.3]}
\graphicspath{{Screenshots/}{../Screenshots}}
\makeatletter
\def\pictext#1#2{%
    \@ifnextchar[{%
    \pictext@iiiii{#1}{#2}%
    }{%
      \pictext@iiiii{#1}{#2}[0.5,0.4,0.3]% Default is 5
    }%
}
\def\pictext@iiiii#1#2[#3,#4,#5]{\begin{minipage}{#3\textwidth}\includegraphics[scale=#4]{#1}\end{minipage}\begin{minipage}{#5\textwidth}#2\end{minipage}}
\def\minipg#1#2{%
    \@ifnextchar[{%
    \minipg@iiii{#1}{#2}%
    }{%
      \minipg@iiii{#1}{#2}[0.3,0.6]% Default is 5
    }%
}
\def\minipg@iiii#1#2[#3,#4]{\vspace{0.8mm}\begin{minipage}{#3\textwidth}#1\end{minipage}\begin{minipage}{#4\textwidth}#2\end{minipage}{\vspace{0.8mm}}}
\makeatother

%\newenvironment{minty}[2]% environment name
%{% begin code
%  \begin{minipage}{#1}
%  \begin{minted}{#2}
%}%
%{% end code
%  \end{minted}
%  \end{minipage}
%  \end{minty}\ignorespacesafterend
%} 

% Smaller Lists
\usepackage{enumitem}
\setlist[itemize,enumerate]{leftmargin=3mm, labelindent=0mm, labelwidth=1mm, labelsep=1mm, nosep}
\setlist[description]{leftmargin=0mm, nosep}
\setlength{\parindent}{0cm}

% Smaller Titles
\usepackage[explicit]{titlesec}

%% Color Boxes
\newcommand{\sectioncolor}[1]{\colorbox{black!60}{\parbox{0.989\linewidth}{\color{white}#1}}}
\newcommand{\subsectioncolor}[1]{\colorbox{black!50}{\parbox{0.989\linewidth}{\color{white}#1}}}
\newcommand{\subsubsectioncolor}[1]{\colorbox{black!40}{\parbox{0.989\linewidth}{\color{white}#1}}}
\newcommand{\paragraphcolor}[1]{\colorbox{black!30}{\parbox{0.989\linewidth}{\color{white}#1}}}
\newcommand{\subparagraphcolor}[1]{\colorbox{black!20}{\parbox{0.989\linewidth}{\color{white}#1}}}

%% Title Format
\titleformat{\section}{\vspace{0.5mm}\bfseries}{}{0mm}{\sectioncolor{\thesection~#1}}[{\vspace{0.5mm}}]
\titleformat{\subsection}{\vspace{0.5mm}\bfseries}{}{0mm}{\subsectioncolor{\thesubsection~#1}}[{\vspace{0.5mm}}]
\titleformat{\subsubsection}{\vspace{0.5mm}\bfseries}{}{0mm}{\subsubsectioncolor{\thesubsubsection~#1}}[{\vspace{0.5mm}}]
\titleformat{\paragraph}{\vspace{0.5mm}\bfseries}{}{0mm}{\paragraphcolor{\theparagraph~#1}}[{\vspace{0.5mm}}]
\titleformat{\subparagraph}{\vspace{0.5mm}\bfseries}{}{0mm}{\subparagraphcolor{\thesubparagraph~#1}}[{\vspace{0.5mm}}]

%% Title Spacing
\titlespacing{\section}{0mm}{0mm}{0mm}
\titlespacing{\subsection}{0mm}{0mm}{0mm}
\titlespacing{\subsubsection}{0mm}{0mm}{0mm}
\titlespacing{\paragraph}{0mm}{0mm}{0mm}
\titlespacing{\subparagraph}{0mm}{0mm}{0mm}

%% format cells
\usepackage[document]{ragged2e}
\usepackage{array, makecell}
\renewcommand{\arraystretch}{2}
\newcommand{\mc}{\makecell[{{m{1\linewidth}}}]}



\begin{document}
\tableofcontents

\newcommand{\TITLE}{Bsys2}
\newcommand{\AUTHOR}{Fabio Lenherr}
\setcounter{tocdepth}{1}

\section{C}

\subsection{fixed size types}
\begin{itemize}
\item \textcolor{purple}{int8\_t, int16\_t, int32\_t, int64\_t}\newline
  fixed integers with bit count
\item \textcolor{purple}{intmax\_t} max size int on platform
\item \textcolor{purple}{intptr\_t} signed integer with the size of an address on this platform
\item \textcolor{purple}{uint8\_t, uintptr\_t} unsigned versions
\item \textcolor{purple}{size\_t} \newline
  this is used in containers, the reson for this is that \emph{this has the max size that for example an array can be.}\newline
  \textcolor{orange}{This is unsigned!}
\end{itemize}

\subsection{Addition of pointers}
If you try to add or subtract 2 pointers to get the amount of sizeof(t) difference, then you can only do this with the exact same type, something like signed int and unsigned int will not work!\newline
\begin{lstlisting}
int32_t *y = 100;
int32_t *x = 120;
ptrdiff_t z = x - y; // z == 5
uint32_t *u = 120;  
ptrdiff_t p = u - y;  // Error: Different ptr types
\end{lstlisting}

\subsection{Index Operator on Pointers}
You can index on pointers like an array, this can be used to get elements on any object.\newline
Note that you have to manually make sure to stay within the bounds of that object, as otherwise you will have \emph{undefined behavior}.\newline
\begin{lstlisting}
int32_t x = 0;
int32_t *y = &x;
y[0] = 0x42;      // same as: x = 0x42;
(&x)[0] = 0x42;   // same
0[&x] = 0x42;     // same
100[200] = 0x42;  // Error: no address
\end{lstlisting}

\subsection{Padding}
When you mix and match different types of different sizes inside of a struct, then the compiler will include padding based on the bigger type: \newline
\begin{lstlisting}
struct  {
char c;     // Offset 0
int32_t x;  // Offset 4 --> Padding
char d;     // Offset 8
} t;        // sizeof t == 12
# structure matters!!
struct  {
char c;     // Offset 0
char d;     // Offset 1
int32_t x;  // Offset 2 --> Padding
} t;        // sizeof t == 6
\end{lstlisting}

\subsection{Forwards Declaration}
\begin{lstlisting}
struct Folder;
// Forward-Deklaration
struct File {
struct Folder *parent;
// OK: all pointer types
//
have same size
char name[256];
// OK: fixed size array
};
// --> Type complete
struct Folder {
struct File * file[256]; // OK: fixed size array
};
// --> Type complete
\end{lstlisting}

\section{Filesystems}

\subsection{Ext2}

\subsection{Ext4}

\section{Processmodels}

\section{Communication and Synchronization}

\section{Programs and libraries}

\section{Graphical Overlays}

\end{document}
